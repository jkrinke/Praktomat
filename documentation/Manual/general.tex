
\section{Some initial remarks}

This documentation is for Praktomat in the new rewritten version.

\subsection{Origins}

The original version of Praktomat was developed in 1998 at the University of Passau by Andreas Zeller. In the following years it was extended by various authors. In 2009, the development and maintenance of this version has been stopped.

In 2011 a new rewritten and modern version has been implemented at the University of Karlsruhe.

Praktomat is still in active use and development at the University of Karlsruhe and a german description of the Praktomat project can be found at \url{http://pp.info.uni-karlsruhe.de/projects/praktomat/praktomat.php}. Praktomat is open source software, hosted on github at \url{https://github.com/KITPraktomatTeam/Praktomat/}.

\subsection{Terms used}

Although the lanuage used in Praktomat is english, there are a few terms that are specific for the University of Karlsruhe and have not yet been changed into more terms used in british or american universities.

\subsubsection{Student}

A student is somebody who submits solutions to Praktomat.

\subsubsection{Trainer}

\subsubsection{Tutor}

\subsubsection{Attestation}

In british universities this would mean grading or marking.